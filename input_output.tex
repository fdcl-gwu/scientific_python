\documentclass[11pt,professionalfonts]{beamer}
\usefonttheme{serif}

\usepackage{presentation_packages}
\bibliography{library} % must be in the preamble when using biblatex package

\definecolor{mygray}{gray}{0.9}
\definecolor{RoyalBlue}{rgb}{0.25,0.41,0.88}
\def\Emph{\textcolor{RoyalBlue}}

\definecolor{tmp}{rgb}{0.804,0.941,1.0}
\setbeamercolor{numerical}{fg=black,bg=tmp}
\setbeamercolor{exact}{fg=black,bg=red}

\mode<presentation> 
{
  \usetheme{Warsaw}
  \usefonttheme{serif}
  \setbeamercovered{transparent}
}

\setbeamertemplate{footline}%{split theme}
{%
  \leavevmode%
  \hbox{\begin{beamercolorbox}[wd=.5\paperwidth,ht=2.5ex,dp=1.125ex,leftskip=.3cm,rightskip=.3cm plus1fill]{author in head/foot}%
    \usebeamerfont{author in head/foot}\insertshorttitle
  \end{beamercolorbox}%
  \begin{beamercolorbox}[wd=.5\paperwidth,ht=2.5ex,dp=1.125ex,leftskip=.3cm,rightskip=.3cm]{title in head/foot}
%    \usebeamerfont{title in head/foot}\mypaper\hfill \insertframenumber/\inserttotalframenumber
    \usebeamerfont{title in head/foot}\hfill \insertframenumber/\inserttotalframenumber
  \end{beamercolorbox}}%
  \vskip0pt%
} \setbeamercolor{box}{fg=black,bg=yellow}


\title[Input/Output]{\large \textbf{Reading/Writing Data}}

\author{\vspace*{-0.3cm}}

   
\institute{
  \footnotesize
  {\normalsize\bf{Shankar Kulumani}}\\
  \vspace*{0.2cm}
    \textbf{Flight Dynamics \& Control Lab}\\ \vspace*{0.5cm}
  \begin{figure} %figure%
        \includegraphics[width=0.75\textwidth]{figures/gw_txh_2cs_pos.pdf}
    \end{figure}
}
\date{}

\begin{document}
%=======================================================%

\setcounter{framenumber}{-1}
\begin{frame} %-----------------------------%
  \titlepage
\end{frame}   %-----------------------------%

\section*{}
\subsection*{Reading Data}  
\begin{frame}{Why import/export?}
\begin{itemize}
    \item Most data will not be ``hard-coded'' into your programs
        \begin{itemize}
            \item Data from observations to process
            \item Data from models to export and analyze
        \end{itemize}
    \item Allow for multiuse programs by reading in data from the user - interaction
\end{itemize}
\end{frame}

\begin{frame}[fragile]{Getting data from the user}
    \begin{itemize}
        \item Prompt the user to input data
        \begin{verbatim}
        a = input("What is the first number? ")
        b = input("What is the second number? ")

        a = int(a)
        b = int(b)
        \end{verbatim}
    \item Input data is converted to a string - you can convert it to a different data type as desired
    \end{itemize}
\end{frame}

\begin{frame}[fragile]{Importing text from files}
    \begin{itemize}
        \item First must ``open'' the file for reading
\begin{verbatim}
file = open("filename.txt", "r")
\end{verbatim}
        \item There are many ways to read a text file in Python
            \begin{itemize}
                \item Reading data by the line - Useful for sequential data (tracking station data)
\begin{verbatim}
print(file.readline()) 
\end{verbatim}
                \item Read a big block of data - Processing data for analysis (missile intercept data)
\begin{verbatim}
print(file.read())
\end{verbatim}
    
            \end{itemize}
        \item After reading we must ``close'' the file
            \begin{verbatim}
file.close()
            \end{verbatim}
    \end{itemize}
\end{frame}

\begin{frame}[fragile]{Opening a file}
    \begin{itemize}
        \item Before reading/writing to a file it must be opened
        \item This will create a file object which we can then operate on
            \begin{verbatim}
file_object = open("filename", "mode")
            \end{verbatim}
        \item Several ways to open a file
            \begin{itemize}
                \item \texttt{'r'} - Read mode which the file is only read
                \item \texttt{'w'} - Write mode which is used to edit and write new data - will erase whatever is in \texttt{"filename"}
                \item \texttt{'a'} - Append mode which is used to add data to the end of the file
                \item \texttt{'r+'} - Read and write mode - both combined into one
            \end{itemize}
    \end{itemize}
\end{frame}


\begin{frame}[fragile]{Reading by line}
    \begin{itemize}
        \item Can use a loop to read a file line by line and print the output
            \begin{verbatim}
file = open("testfile.txt", 'r')
for line in file:
    print(line)
file.close()
            \end{verbatim}
        \item Convienent to use the \texttt{with} statement
            \begin{verbatim}
with open("testfile.txt", 'r') as file:
    for line in file:
        print(line)
            \end{verbatim}
        \item \texttt{with} will automatically close the file for us
    \end{itemize}
\end{frame}

\begin{frame}[fragile]{Exporting data}
    \begin{itemize}
        \item After performing calculations we'd like to have a way to save the results to a file for later use
        \item Two different ways to save file - among many, many others
            \begin{itemize}
                \item Write to text files
                \item Save to a \texttt{numpy} file 
                \item Save to a \texttt{*.dat} file (in case you want to read the data in Matlab later) 
                % This was helpful to me when I was working with Mahdis since she had her whole code 
                % already written in Matlab. It was much convenient for her to read from a dat file 
                % rather than writing a new function to read from a text file.
            \end{itemize}
    \end{itemize}
\end{frame}

\begin{frame}[fragile]{Writing text to a file}
    \begin{itemize}
        \item Can write strings to a text file
            \begin{verbatim}
with open("textfile.txt", 'a') as file:
    file.write("This is a string")
    file.write("Here is another string")
            \end{verbatim}
    \end{itemize}
\end{frame}

\begin{frame}[fragile]{Writing to a data file}
    \begin{itemize}
        \item We can save data ``as-is'' for later use
        \item Preserve arrays/lists etc
            \begin{verbatim}
import numpy as np
a = np.arange(10)
b = np.arange(100)
np.save("outfile", (a, b))
            \end{verbatim}
    \end{itemize}
\end{frame}

\begin{frame}{Exercise: Import and export data}
    \begin{itemize}
        \item There is a file called \texttt{vector.txt} 
            \begin{itemize}
                \item 6 numbers which represent 3 position and 3 velocity components all in a single line
            \end{itemize}
        \item Write a function \texttt{getposvel} that will:
            \begin{itemize}
                \item Import the first three numbers into a position array
                \item Import last three numbers into a velocity array
                \item Inputs: filename to open, Outputs: \texttt{pos, vel} the output arrays
            \end{itemize}
    \end{itemize}
\end{frame}

\begin{frame}[fragile]{Exercise: Testing}
    \begin{itemize}
        \item Your code should be tested. 
            It should at a minimum pass the following unit tests
    \begin{verbatim}
def test_getposvec_array_sizes():
    pos, vel = getposvec('vector.txt')
    np.testing.assert_allclose(pos.shape, (3,))
    np.testing.assert_allclose(vel.shape, (3,))
    \end{verbatim}
    \item Determine what other tests you need to verify your function
    \end{itemize}
\end{frame}

\begin{frame}{Exercise: Output}
    \begin{itemize}
        \item Write a function \texttt{writevec} that will:
            \begin{itemize}
                \item Create a \texttt{*.txt} file with the user's desired name
                \item Label the components of a vector with the user's choice 
                \item Write the components of a vector, with label to a the text file (x, y, z or i, j, k, etc)
                \item Inputs: filename, vector, array of labels
                \item Outputs: Success or failure flag
            \end{itemize}
    \end{itemize}
\end{frame}
\end{document}

