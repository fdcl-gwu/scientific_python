\documentclass[11pt,professionalfonts]{beamer}
\usefonttheme{serif}

\usepackage{presentation_packages}
\bibliography{library} % must be in the preamble when using biblatex package

\definecolor{mygray}{gray}{0.9}
\definecolor{RoyalBlue}{rgb}{0.25,0.41,0.88}
\def\Emph{\textcolor{RoyalBlue}}

\definecolor{tmp}{rgb}{0.804,0.941,1.0}
\setbeamercolor{numerical}{fg=black,bg=tmp}
\setbeamercolor{exact}{fg=black,bg=red}

\mode<presentation> 
{
  \usetheme{Warsaw}
  \usefonttheme{serif}
  \setbeamercovered{transparent}
}

\setbeamertemplate{footline}%{split theme}
{%
  \leavevmode%
  \hbox{\begin{beamercolorbox}[wd=.5\paperwidth,ht=2.5ex,dp=1.125ex,leftskip=.3cm,rightskip=.3cm plus1fill]{author in head/foot}%
    \usebeamerfont{author in head/foot}\insertshorttitle
  \end{beamercolorbox}%
  \begin{beamercolorbox}[wd=.5\paperwidth,ht=2.5ex,dp=1.125ex,leftskip=.3cm,rightskip=.3cm]{title in head/foot}
%    \usebeamerfont{title in head/foot}\mypaper\hfill \insertframenumber/\inserttotalframenumber
    \usebeamerfont{title in head/foot}\hfill \insertframenumber/\inserttotalframenumber
  \end{beamercolorbox}}%
  \vskip0pt%
} \setbeamercolor{box}{fg=black,bg=yellow}


\title[Lesson 05 - Structure]{\large \textbf{Structuring your Code}}

\author{\vspace*{-0.3cm}}

   
\institute{
  \footnotesize
  {\normalsize\bf{Shankar Kulumani}}\\
  \vspace*{0.2cm}
    \textbf{Flight Dynamics \& Control Lab}\\ \vspace*{0.5cm}
  \begin{figure} %figure%
        \includegraphics[width=0.75\textwidth]{figures/gw_txh_2cs_pos.pdf}
    \end{figure}
}
\date{}

\begin{document}
%=======================================================%

\setcounter{framenumber}{-1}
\begin{frame} %-----------------------------%
  \titlepage
\end{frame}   %-----------------------------%

\section*{}
\subsection*{Structure}  

\begin{frame}{Structure of Python Code}
    \begin{itemize}
        \item Your Python code is made up of several components
            \begin{itemize}
                \item Lines of variables/evaluations
                \item Functions - several lines which achieve a single purpose
                \item Modules - several related functions (file) 
                \item Packages - several related modules (directory)
            \end{itemize}
        \item It helps to create a standard structure so your code is organized and easy to use and modify
        \item Also much of the automated testing and other features depend on a clear structure for you code
    \end{itemize}
\end{frame}

\begin{frame}{Python is easy}
    \centering
    \includegraphics[height=0.8\textheight]{figures/python_xkcd.png}
\end{frame}

\begin{frame}[fragile]\frametitle{Basic Structure Example}
    \begin{itemize}
        \item The \texttt{astro} library is a good example of the basic structure
    \end{itemize}
    \begin{verbatim}
        README.md
        driver.py
        package_a
            __init__.py
            module_a.py
            module_b.py
        package_b
            __init__.py
            module_a.py
        tests
            __init__.py
            test_package_a.py
            test_package_b.py
    \end{verbatim}
\end{frame}

\begin{frame}{Structure of Code is Key}
    \begin{itemize}
        \item Avoid Circular dependencies - modules and packages should try to be independent
        \item Hidden coupling - making a small change breaks many other functions/tests
        \item Passing variables instead of globals - many objects can modify a global variable
    \end{itemize}
\end{frame}

\begin{frame}{Modules}
    \begin{itemize}
        \item One of the main abstraction layers - separate code into parts holding related data and functionality
            \begin{itemize}
                \item One layer might handle user input 
                \item Another layer handles low-level data manipulation
                \item Group each into separate files
                \item Python \texttt{import} and \texttt{from ... import} statements
            \end{itemize}
        \item Module names should be short, lowercase, and avoid special symbols
            \begin{itemize}
                \item Nothing else special is required - it's just a file with a fancy name now
                \item Python will search for the file - and make it available using the \texttt{import} statement
            \end{itemize}
    \end{itemize}


\end{frame}

\begin{frame}[fragile]\frametitle{Best Practices}
Very Bad
\begin{verbatim}
[...]
from modu import *
[...]
x = sqrt(4) # where did sqrt come from?
\end{verbatim}
\pause
Better
\begin{verbatim}
from modu import sqrt
[...]
x = sqrt(4) # sqrt might be part of modu, if not redefined inbetween
\end{verbatim}
\pause
Best
\begin{verbatim}
import modu
[...]
x = modu.sqrt(4) # sqrt is visibly part of modu
\end{verbatim}

\pause
\begin{alertblock}{}
    \centering
Readability and Clarity are important
\end{alertblock}
\end{frame}

\begin{frame}[fragile]\frametitle{Packages}
    \begin{itemize}
        \item Packages are a simple extension to modules
        \item Just place an empty file named \texttt{\_\_init\_\_.py} in a directory
        \item Same import structure as modules
    \end{itemize}

    Convenient syntax for deeply nested packages
    \begin{verbatim}
        import very.deep.module as mod
    \end{verbatim}
\end{frame}

\end{document}

