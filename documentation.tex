\documentclass[11pt,professionalfonts]{beamer}
\usefonttheme{serif}

\usepackage{presentation_packages}
\bibliography{library} % must be in the preamble when using biblatex package
\usepackage{fancyvrb}
\fvset{fontsize=\normalsize}

\RecustomVerbatimEnvironment{verbatim}{Verbatim}{}

\definecolor{mygray}{gray}{0.9}
\definecolor{RoyalBlue}{rgb}{0.25,0.41,0.88}
\def\Emph{\textcolor{RoyalBlue}}

\definecolor{tmp}{rgb}{0.804,0.941,1.0}
\setbeamercolor{numerical}{fg=black,bg=tmp}
\setbeamercolor{exact}{fg=black,bg=red}

\mode<presentation> 
{
  \usetheme{Warsaw}
  \usefonttheme{serif}
  \setbeamercovered{transparent}
}

\setbeamertemplate{footline}%{split theme}
{%
  \leavevmode%
  \hbox{\begin{beamercolorbox}[wd=.5\paperwidth,ht=2.5ex,dp=1.125ex,leftskip=.3cm,rightskip=.3cm plus1fill]{author in head/foot}%
    \usebeamerfont{author in head/foot}\insertshorttitle
  \end{beamercolorbox}%
  \begin{beamercolorbox}[wd=.5\paperwidth,ht=2.5ex,dp=1.125ex,leftskip=.3cm,rightskip=.3cm]{title in head/foot}
%    \usebeamerfont{title in head/foot}\mypaper\hfill \insertframenumber/\inserttotalframenumber
    \usebeamerfont{title in head/foot}\hfill \insertframenumber/\inserttotalframenumber
  \end{beamercolorbox}}%
  \vskip0pt%
} \setbeamercolor{box}{fg=black,bg=yellow}


\title[Lesson 03 - Documentation]{\large \textbf{Documenting Your Code}}

\author{\vspace*{-0.3cm}}

   
\institute{
  \footnotesize
  {\normalsize\bf{Shankar Kulumani}}\\
  \vspace*{0.2cm}
    \textbf{Flight Dynamics \& Control Lab}\\ \vspace*{0.5cm}
  \begin{figure} %figure%
        \includegraphics[width=0.75\textwidth]{figures/gw_txh_2cs_pos.pdf}
    \end{figure}
}
\date{}

\begin{document}
%=======================================================%

\setcounter{framenumber}{-1}
\begin{frame} %-----------------------------%
  \titlepage
\end{frame}   %-----------------------------%

\section*{}
\subsection*{Documentation}  
\begin{frame}{What is documentation?}
\begin{itemize}
    \item Documentation is the text commentary in a computer program that describes how it works
    \item Can include any/all of the following:
        \begin{itemize}
            \item Variable definitions/units
            \item General usage information - options, constants
            \item Program flow descriptions
        \end{itemize}
\end{itemize}
\end{frame}

\begin{frame}{Why is documentation important?}
\begin{itemize}
    \item Helps future users (others or a future self) understand how the program works
        \begin{itemize}
            \item Determine how to use the program
            \item Allows for future modifications without having to rewrite everything
        \end{itemize}
    \item For this course you'll have to comply with the Astrodynamic Programming Standard
\end{itemize}
\end{frame}

\begin{frame}{What is the standard for documentation?}
\begin{itemize}
    \item Format - function documentation and inline documentation
    \item Example - example provided in \href{https://github.com/fdcl-gwu/MAE3145_library}{astro library}
    \item Python uses \texttt{\#} or \texttt{"""} (triple double quote) for comments
\end{itemize}
\end{frame}

\begin{frame}[fragile]{Module Documentation Block}
    \begin{verbatim}[fontsize=\scriptsize]
"""Module name

Extended description of the module

Notes
-----
    This is an example of an indented section. It's like any other section,
    but the body is indented to help it stand out from surrounding text.

If a section is indented, then a section break is created by
resuming unindented text.

Attributes
----------
module_level_variable1 : int
    Descrption of the variable

Author
------
Shankar Kulumani		GWU		skulumani@gwu.edu
"""
    \end{verbatim}
\end{frame}

\begin{frame}[fragile]{Function Documentation Block}
    \begin{verbatim}[fontsize=\tiny]
    r"""A one line description of the function

    This function will take two inputs and find the sum. 
    The inputs can be either scalars or numpy arrays of the same size.

    Parameters
    ----------
    a : any int or float  
        First input
    b : any int or float
        Another input

    Returns
    -------
    out : same as input
        Sum of a and b.

    Notes
    -----
    Both inputs should be the same size

    Author
    ------
    Shankar Kulumani		GWU		skulumani@gwu.edu

    Examples
    --------
    An example of how to use the function

    >>> a = [1, 2, 3]
    >>> b = [2, 3, 4]
    >>> out = add_two_inputs(a, b)
    out = [3, 5, 7]

    """
    \end{verbatim}
\end{frame}
\begin{frame}{Documentation Standards}
    \begin{itemize}
        \item Documentation Block at beginning of every function and module
            \begin{itemize}
                \item Inputs and Outputs
                \item Example usage
                \item Coupling of other functions
            \end{itemize}
        \item Inline documentation
            \begin{itemize}
                \item General comments on why you're doing something (not what) - we can just read the code
                    \begin{itemize}
                        \item Bad - \texttt{ sum = a + b \# we add two numbers}
                        \item Good - \texttt{sum = a + b \# need to sum to get an average later on}
                    \end{itemize}
                \item Use whitespace to make to code readable 
            \end{itemize}
    \end{itemize}
\end{frame}

\end{document}

