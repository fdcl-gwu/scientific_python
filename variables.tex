\documentclass[11pt,professionalfonts]{beamer}
\usefonttheme{serif}

\usepackage{presentation_packages}
\bibliography{library} % must be in the preamble when using biblatex package

\definecolor{mygray}{gray}{0.9}
\definecolor{RoyalBlue}{rgb}{0.25,0.41,0.88}
\def\Emph{\textcolor{RoyalBlue}}

\definecolor{tmp}{rgb}{0.804,0.941,1.0}
\setbeamercolor{numerical}{fg=black,bg=tmp}
\setbeamercolor{exact}{fg=black,bg=red}

\mode<presentation> 
{
  \usetheme{Warsaw}
  \usefonttheme{serif}
  \setbeamercovered{transparent}
}

\setbeamertemplate{footline}%{split theme}
{%
  \leavevmode%
  \hbox{\begin{beamercolorbox}[wd=.5\paperwidth,ht=2.5ex,dp=1.125ex,leftskip=.3cm,rightskip=.3cm plus1fill]{author in head/foot}%
    \usebeamerfont{author in head/foot}\insertshorttitle
  \end{beamercolorbox}%
  \begin{beamercolorbox}[wd=.5\paperwidth,ht=2.5ex,dp=1.125ex,leftskip=.3cm,rightskip=.3cm]{title in head/foot}
%    \usebeamerfont{title in head/foot}\mypaper\hfill \insertframenumber/\inserttotalframenumber
    \usebeamerfont{title in head/foot}\hfill \insertframenumber/\inserttotalframenumber
  \end{beamercolorbox}}%
  \vskip0pt%
} \setbeamercolor{box}{fg=black,bg=yellow}


\title[Variables]{\large \textbf{Variables in Python}}

\author{\vspace*{-0.3cm}}

   
\institute{
  \footnotesize
  {\normalsize\bf{Shankar Kulumani}}\\
  \vspace*{0.2cm}
    \textbf{Flight Dynamics \& Control Lab}\\ \vspace*{0.5cm}
  \begin{figure} %figure%
        \includegraphics[width=0.75\textwidth]{figures/gw_txh_2cs_pos.pdf}
    \end{figure}
}
\date{}

\begin{document}
%=======================================================%

\setcounter{framenumber}{-1}
\begin{frame} %-----------------------------%
  \titlepage
\end{frame}   %-----------------------------%

\section*{}
\subsection*{Variables}  
\begin{frame}{What is a variable?}
    \begin{itemize}
        \item Assignment or placeholder in memory 
        \item Variables have names
        \item Variables can be the basis of operations
        \item Variables can be assigned in the interpreter or a script file
    \end{itemize}
\end{frame}

\begin{frame}[fragile]\frametitle{frametitle}
    \begin{itemize}
        \item Variables can be scalars or matrices
    \end{itemize}
\begin{verbatim}
    import numpy
    a = 3.5
    amatrix = numpy.array([[1, 2], [3, 4]])
\end{verbatim}
\end{frame}

\begin{frame}[fragile]\frametitle{Vectors}
    \begin{itemize}
        \item Vectors can be 1 or 2 dimensional
        \item Best practice is to use 1 dimensional, Numpy will figure out the rest
    \end{itemize}
\begin{verbatim}
import numpy
a = numpy.array([1, 2, 3])
b = numpy.array([[1, 2, 3]])
c = numpy.array([[1], [2], [3]])
\end{verbatim}

\end{frame}

\begin{frame}{Where do variables reside?}
    \begin{itemize}
        \item Functions - visible inside the function
        \item Modules - entire module can access (and outside using \texttt{import})
        \item Scripts/Interpreter - Only the script/interactive can view the variables
    \end{itemize}
\end{frame}

\begin{frame}{Naming}
    \begin{itemize}
        \item Use variable names which actually match the data they hold
        \item Names are case sensitive
        \item One good approach is lowercase with underscores
    \end{itemize}
\end{frame}

\begin{frame}[fragile]\frametitle{Indexing in Python}
    \begin{itemize}
        \item Arrays are zero based
        \item Can slice/view parts of a big array easily
    \end{itemize}
    \begin{verbatim}
    import numpy
    A = numpy.reshape(numpy.arange(25), (5, 5))
    A[3, 4]
    A[:, 1]
    A[1:3,4]
    A[2:, 0]
    \end{verbatim}
\end{frame}

\begin{frame}[fragile]\frametitle{Indexing}
    \begin{itemize}
        \item Find the size of an array
    \begin{verbatim}
    A.shape
    numpy.shape(A)
    \end{verbatim}
\item Modify items in an array with slicign
    \begin{verbatim}
    A[:,0] = [100, 200, 300, 400, 500]
    \end{verbatim}
    \end{itemize}
\end{frame}
\end{document}

