\documentclass[11pt,professionalfonts]{beamer}
\usefonttheme{serif}

\usepackage{presentation_packages}
\bibliography{library} % must be in the preamble when using biblatex package

\definecolor{mygray}{gray}{0.9}
\definecolor{RoyalBlue}{rgb}{0.25,0.41,0.88}
\def\Emph{\textcolor{RoyalBlue}}

\definecolor{tmp}{rgb}{0.804,0.941,1.0}
\setbeamercolor{numerical}{fg=black,bg=tmp}
\setbeamercolor{exact}{fg=black,bg=red}

\mode<presentation> 
{
  \usetheme{Warsaw}
  \usefonttheme{serif}
  \setbeamercovered{transparent}
}

\setbeamertemplate{footline}%{split theme}
{%
  \leavevmode%
  \hbox{\begin{beamercolorbox}[wd=.5\paperwidth,ht=2.5ex,dp=1.125ex,leftskip=.3cm,rightskip=.3cm plus1fill]{author in head/foot}%
    \usebeamerfont{author in head/foot}\insertshorttitle
  \end{beamercolorbox}%
  \begin{beamercolorbox}[wd=.5\paperwidth,ht=2.5ex,dp=1.125ex,leftskip=.3cm,rightskip=.3cm]{title in head/foot}
%    \usebeamerfont{title in head/foot}\mypaper\hfill \insertframenumber/\inserttotalframenumber
    \usebeamerfont{title in head/foot}\hfill \insertframenumber/\inserttotalframenumber
  \end{beamercolorbox}}%
  \vskip0pt%
} \setbeamercolor{box}{fg=black,bg=yellow}


\title[Algorithms]{\large \textbf{Structred Programming}}

\author{\vspace*{-0.3cm}}

   
\institute{
  \footnotesize
  {\normalsize\bf{Shankar Kulumani}}\\
  \vspace*{0.2cm}
    \textbf{Flight Dynamics \& Control Lab}\\ \vspace*{0.5cm}
  \begin{figure} %figure%
        \includegraphics[width=0.75\textwidth]{figures/gw_txh_2cs_pos.pdf}
    \end{figure}
}
\date{}

\begin{document}
%=======================================================%

\setcounter{framenumber}{-1}
\begin{frame} %-----------------------------%
  \titlepage
\end{frame}   %-----------------------------%

\section*{}
\subsection*{Structured Programming}  
\begin{frame}{What is structured programming?}
    \begin{itemize}
        \item Concerned with using module functions or sub-programs
        \item Each function has a speciality
        \item The main program/function has very little code - simply calls other functions
    \end{itemize}
\end{frame}

\begin{frame}{Why use structured programming?}
    \begin{itemize}
        \item Debugging - can test small components instead of a giant program all at once
        \item Maintainability - easy to add new features, others can help, others can understand 
        \item Reuse - can use functions in other programs - reduce repetition
        \item 
    \end{itemize}
\end{frame}

\begin{frame}{Algorithms}
    \begin{itemize}
        \item A logical sequence of steps that describes in detail the process used to solve a given problem
        \item Similar to a flow chart or diagram
        \item First come up with a plan before trying to blindly write code
    \end{itemize}
   
    \pause
    \begin{alertblock}{Key Idea}
        \centering
        You should be able to give the algorithm to any programmer, without any astrodynamics or engineering background, and they should be able to write your program
    \end{alertblock}
\end{frame}

\begin{frame}{Examples}
    \begin{itemize}
        \item Write a structured algorithm to read in from a file a series of \(x, y\) pairs for 2D vectors.
            Compute the magnitude of each and print to an output file
        \item You must have a algorithm for the main program and any sub functions
    \end{itemize}
\end{frame}
\end{document}

