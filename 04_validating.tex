\documentclass[11pt,professionalfonts]{beamer}
\usefonttheme{serif}

\usepackage{presentation_packages}
\bibliography{library} % must be in the preamble when using biblatex package

\definecolor{mygray}{gray}{0.9}
\definecolor{RoyalBlue}{rgb}{0.25,0.41,0.88}
\def\Emph{\textcolor{RoyalBlue}}

\definecolor{tmp}{rgb}{0.804,0.941,1.0}
\setbeamercolor{numerical}{fg=black,bg=tmp}
\setbeamercolor{exact}{fg=black,bg=red}

\mode<presentation> 
{
  \usetheme{Warsaw}
  \usefonttheme{serif}
  \setbeamercovered{transparent}
}

\setbeamertemplate{footline}%{split theme}
{%
  \leavevmode%
  \hbox{\begin{beamercolorbox}[wd=.5\paperwidth,ht=2.5ex,dp=1.125ex,leftskip=.3cm,rightskip=.3cm plus1fill]{author in head/foot}%
    \usebeamerfont{author in head/foot}\insertshorttitle
  \end{beamercolorbox}%
  \begin{beamercolorbox}[wd=.5\paperwidth,ht=2.5ex,dp=1.125ex,leftskip=.3cm,rightskip=.3cm]{title in head/foot}
%    \usebeamerfont{title in head/foot}\mypaper\hfill \insertframenumber/\inserttotalframenumber
    \usebeamerfont{title in head/foot}\hfill \insertframenumber/\inserttotalframenumber
  \end{beamercolorbox}}%
  \vskip0pt%
} \setbeamercolor{box}{fg=black,bg=yellow}


\title[Lesson 03 - Validation]{\large \textbf{Validating Your Code}}

\author{\vspace*{-0.3cm}}

   
\institute{
  \footnotesize
  {\normalsize\bf{Shankar Kulumani}}\\
  \vspace*{0.2cm}
    \textbf{Flight Dynamics \& Control Lab}\\ \vspace*{0.5cm}
  \begin{figure} %figure%
        \includegraphics[width=0.75\textwidth]{figures/gw_txh_2cs_pos.pdf}
    \end{figure}
}
\date{}

\begin{document}
%=======================================================%

\setcounter{framenumber}{-1}
\begin{frame} %-----------------------------%
  \titlepage
\end{frame}   %-----------------------------%

\section*{}
\subsection*{Validation}  
\begin{frame}{What do I validate that a program is working?}
    \begin{itemize}
        \item When your program runs without a syntax error, you might be so relieved, that you won't try and consider if the results are actually correct
        \item There must always be a method to provide confidence that the output is what you expect
        \item A wrong answer might be worse than no answer - Many, Many examples of software errors causing mission failures!
    \end{itemize}
\end{frame}

\begin{frame}{Determining test cases}
    \begin{itemize}
        \item Come up with a test case that you have indepently determined answer by another method
            \begin{itemize}
                \item Hand calculations
                \item Examples from a textbook (make sure it's actually correct first!)
                \item Another well-trusted/validated program (its not about matching someone else's answer but actually the correct answer!)
            \end{itemize}
        \item Compare the output of your program with the test case and make sure it matches
            \begin{itemize}
                \item One good approach is to come up with all of the test cases first, before even writing your program.
                \item Write your program to pass all of the test cases
            \end{itemize}
        \item You will usually need many test cases for your code
            \begin{itemize}
                \item Every branch/line of code should be evaluated through all of your test cases
                \item For \texttt{if} statements you'll need a case for each branch
            \end{itemize}
    \end{itemize}
\end{frame}

\begin{frame}{Unit Testing}
    \begin{itemize}
        \item The process of testing small components of code is called \texttt{unit testing}
            \begin{itemize}
                \item We test each \texttt{unit} independently
                \item Easier to test small parts of code rather than somthing huge and complicated
                \item After all units are tested we can then move onto building larger programs
            \end{itemize}
        \item Testing should be fast and automatic
            \begin{itemize}
                \item Writing software is a continual process - never ends
                \item Testing can make sure that future modifications don't break old code 
                \item If something does break, we can easily find the error and correct it
            \end{itemize} 
    \end{itemize}
\end{frame}

\begin{frame}{Testing Example}
    \begin{itemize}
        \item Now we'll practice unit testing in Python
    \end{itemize}
\end{frame}
\end{document}

