\documentclass[11pt,professionalfonts]{beamer}
\usefonttheme{serif}

\usepackage{presentation_packages}
\bibliography{library} % must be in the preamble when using biblatex package

\definecolor{mygray}{gray}{0.9}
\definecolor{RoyalBlue}{rgb}{0.25,0.41,0.88}
\def\Emph{\textcolor{RoyalBlue}}

\definecolor{tmp}{rgb}{0.804,0.941,1.0}
\setbeamercolor{numerical}{fg=black,bg=tmp}
\setbeamercolor{exact}{fg=black,bg=red}

\mode<presentation> 
{
  \usetheme{Warsaw}
  \usefonttheme{serif}
  \setbeamercovered{transparent}
}

\setbeamertemplate{footline}%{split theme}
{%
  \leavevmode%
  \hbox{\begin{beamercolorbox}[wd=.5\paperwidth,ht=2.5ex,dp=1.125ex,leftskip=.3cm,rightskip=.3cm plus1fill]{author in head/foot}%
    \usebeamerfont{author in head/foot}\insertshorttitle
  \end{beamercolorbox}%
  \begin{beamercolorbox}[wd=.5\paperwidth,ht=2.5ex,dp=1.125ex,leftskip=.3cm,rightskip=.3cm]{title in head/foot}
%    \usebeamerfont{title in head/foot}\mypaper\hfill \insertframenumber/\inserttotalframenumber
    \usebeamerfont{title in head/foot}\hfill \insertframenumber/\inserttotalframenumber
  \end{beamercolorbox}}%
  \vskip0pt%
} \setbeamercolor{box}{fg=black,bg=yellow}


\title[Integration]{\large \textbf{Numerical integration in Python}}

\author{\vspace*{-0.3cm}}

   
\institute{
  \footnotesize
  {\normalsize\bf{Shankar Kulumani}}\\
  \vspace*{0.2cm}
    \textbf{Flight Dynamics \& Control Lab}\\ \vspace*{0.5cm}
  \begin{figure} %figure%
        \includegraphics[width=0.75\textwidth]{figures/gw_txh_2cs_pos.pdf}
    \end{figure}
}
\date{}

\begin{document}
%=======================================================%

\setcounter{framenumber}{-1}
\begin{frame} %-----------------------------%
  \titlepage
\end{frame}   %-----------------------------%

\section*{}
\subsection*{Integration}  
\begin{frame}{Integration}
    \begin{itemize}
        \item To find out the future position of a satellite we can  integrate the equations of motion
    \end{itemize}
    \begin{align}
        \ddot{r} &= -\frac{\mu}{r^2} \hat{r} \\
        \dot{r} &= \int \ddot{r} dt \\
        r &= \int \dot{r} dt
    \end{align}
\end{frame}

\begin{frame}{Integration}
    \begin{itemize}
        \item For many problems it's very difficult/impossible to find an analytical solution
        \item We instead use numerical integration rather than trying to find the analytical solution (but we have the analytical solution for some cases of astrodynamics)
    \end{itemize}
\end{frame}

\begin{frame}{Numerical Integration}
    
\end{frame}
\end{document}

